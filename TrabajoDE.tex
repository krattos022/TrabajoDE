\documentclass[12pt,letterpaper]{report}
\usepackage[utf8]{inputenc}
\usepackage[spanish]{babel}
\usepackage{amsmath}
\usepackage{amsfonts}
\usepackage{amssymb}
\usepackage{graphicx}
\usepackage[left=3cm,right=2.5cm,top=3cm,bottom=2cm]{geometry}
\author{Fernando Manzanares}
\title{Diseños de Experimentos en R}
\renewcommand{\baselinestretch}{1.5}
\setlength{\parskip}{4mm}
\usepackage{ragged2e}
\justifying
\usepackage{longtable}
\usepackage{Sweave}
\setlength\parindent{0pt}
\usepackage{titlesec} 
\usepackage{xcolor}
\titleformat{\chapter}[display]{\bfseries\Huge}{\Huge\thechapter.}{0.5em}{}
\titlespacing{\chapter}{0pt}{0ex}{1pc}
\titleformat{\section}[display]{\bfseries\Huge}{\Huge\thechapter.}{0.5em}{}
\titlespacing{\section}{0pt}{0ex}{1pc}
\usepackage{upgreek}
\widowpenalty=10000  
\clubpenalty=10000
\usepackage{multirow}

\begin{document}
\Sconcordance{concordance:TrabajoDE.tex:TrabajoDE.Rnw:%
1 29 1}
\Sconcordance{concordance:TrabajoDE.tex:./Portada2.Rnw:ofs 30:%
1 34 1}
\Sconcordance{concordance:TrabajoDE.tex:./Conceptos.Rnw:ofs 65:%
1 86 1}
\Sconcordance{concordance:TrabajoDE.tex:./D_unifact.Rnw:ofs 152:%
1 73 1 1 2 1 0 1 2 1 0 1 2 1 0 1 1 1 2 1 0 1 3 5 0 1 2 33 1 1 2 1 0 1 1 %
1 4 3 0 1 3 5 0 1 2 3 1 1 2 1 0 2 1 3 0 1 2 60 1 1 7 6 0 5 1 3 0 1 2 24 %
1 1 2 1 0 1 1 1 4 3 0 1 3 5 0 1 2 3 1 1 2 1 0 7 1 3 0 1 2 45 1 1 2 1 0 %
2 1 3 0 1 2 59 1 1 2 1 0 1 3 2 0 3 1 4 0 1 3 20 1 1 2 1 0 1 1 1 4 3 0 1 %
3 5 0 1 2 3 1 1 2 1 0 7 1 3 0 1 2 48 1 1 2 1 0 2 1 3 0 1 2 26 1}
\Sconcordance{concordance:TrabajoDE.tex:./D_Bloques.Rnw:ofs 687:%
1 92 1 1 2 1 0 1 2 1 0 1 2 1 0 1 1 1 2 1 0 1 3 5 0 1 2 24 1 1 2 1 0 1 1 %
1 4 3 0 1 3 5 0 1 2 1 1 1 2 1 0 7 1 3 0 1 2 49 1 1 2 1 0 2 1 3 0 1 2 81 %
1 1 2 1 0 1 2 1 0 1 2 1 0 1 1 1 2 1 0 1 3 5 0 1 2 23 1 1 2 1 0 1 1 1 4 %
3 0 1 3 5 0 1 2 1 1 1 2 1 0 7 1 3 0 1 2 48 1 1 2 1 0 2 1 3 0 1 2 22 1 1 %
2 17 0 1 2 50 1 1 2 1 0 1 4 3 0 1 4 3 0 2 1 3 0 1 2 17 1}
\Sconcordance{concordance:TrabajoDE.tex:./D_factoriales.Rnw:ofs 1233:%
1 147 1 1 2 1 0 1 2 1 0 1 2 1 0 1 1 1 2 1 0 1 3 5 0 1 2 112 1 1 2 1 0 1 %
3 2 0 1 4 3 0 2 1 3 0 1 2 154 1 1 2 1 0 1 2 1 0 1 2 1 0 1 1 1 2 1 0 1 3 %
5 0 1 2 61 1}
\Sconcordance{concordance:TrabajoDE.tex:TrabajoDE.Rnw:ofs 1755:%
36 5 1}

\begin{titlepage}

\begin{center}
\vspace*{-1in}
\begin{figure}[htb]
\begin{center}
\includegraphics[width=6cm]{Minerva}
\end{center}
\end{figure}

FACULTAD MULTIDISCIPLINARIA DE OCCIDENTE\\
\vspace*{0.15in}
DEPARTAMENTO DE MATEMÁTICAS \\
\vspace*{0.6in}
\begin{large}
TITULO:\\
\end{large}
\vspace*{0.2in}
\begin{Large}
\textbf{DISEÑOS DE EXPERIMENTOS EN R} \\
\end{Large}
\vspace*{0.3in}
\begin{large}
Fernando Ernesto Manzanares Morán\\
\end{large}
\vspace*{0.3in}
\rule{80mm}{0.1mm}\\
\vspace*{0.1in}
\begin{large}
Docente: \\
Salvador Enrique Rodriguez Hernandez \\
\end{large}
\end{center}

\end{titlepage}
\chapter*{Definición de un diseño experimental y conceptos básicos}
El \textbf{diseño de un experimento} es la secuencia completa de los pasos que se deben
tomar de antemano, para planear y asegurar la obtención de toda la información relevante y
adecuada al problema bajo investigación, la cual será analizada estadísticamente para obtener
conclusiones válidas y objetivas con respecto a los objetivos planteados.

Un \textbf{diseño experimental} es una prueba o serie de pruebas en las cuales existen
cambios deliberados en las variables de entrada de un proceso o sistema, de tal manera que
sea posible observar e identificar las causas de los cambios que se producen en la respuesta de
salida.

\textbf{Proposito de un diseño experimental}

El propósito de cualquier Diseño Experimental, es proporcionar una cantidad máxima de
información pertinente al problema que se está investigando. Y ajustar el diseño que sea lo
mas simple y efectivo; para ahorrar dinero, tiempo, personal y material experimental que se va
ha utilizar. Es de acotar, que la mayoría de los diseños estadísticos simples, no sólo son fáciles
de analizar, sino también son eficientes en el sentido económico y en el estadístico.
De lo anterior, se deduce que el diseño de un experimento es un proceso que explica
tanto la metodología estadística como el análisis económico.

\textbf{Conceptos básicos}


\textbf{Diseño:} Consiste en planificar la forma de hacer el experimento, materiales y métodos a
usar, etc.

\textbf{Experimento:} Conjunto de pruebas o ensayos cuyo objetivo es obtener 
información, que permita mejorar el producto o el proceso en estudio.

\textbf{Tratamiento:}
- Es un conjunto particular de condiciones experimentales definidas por el investigador; son el conjunto de circunstancias creadas por el experimento, en respuesta a la hipótesis de investigación y son el centro de la misma.

\textbf{Factor:} Es un grupo específico de tratamientos. (Ejemplo, Temperatura, humedad, tipos
de suelos, etc.).

\textbf{Niveles del factor:} Son diversas categorías de un factor. ( Por ejemplo, los niveles
de temperatura son 20°C, 30°C, etc.). Un factor Cuantitativo tiene niveles
asociados con puntos ordenados en alguna escala de medición, como
temperatura; mientras que los niveles de un factor cualitativo representan
distintas categorías o clasificaciones, como tipo de suelo, que no se puede acomodar
conforme a alguna magnitud.

\textbf{Réplica:} Son las repeticiones que se realizan del experimento básico.

\textbf{Unidad experimental:}

Es el material experimental unitario que recibe la aplicación de un tratamiento.

Es la entidad física o el sujeto expuesto al tratamiento independientemente de las
otras unidades. La unidad experimental una vez expuesta al tratamiento constituye
una sola réplica del tratamiento.

Es el objeto o espacio al cual se aplica el tratamiento y donde se mide y analiza la
variable que se investiga.

Es el elemento que se está estudiando.

\textbf{Unidad muestral:} Es una fracción de la unidad experimental que se utiliza 
para medir el efecto de un tratamiento.

\textbf{Error experimental:} Es una medida de variación que existe entre dos o 
más unidades experimentales, que han recibido la aplicación de un mismo tratamiento de
manera idéntica e independiente.

\textbf{Factores controlables:} Son aquellos parámetros o características del 
producto o proceso, para los cuales se prueban distintas variables o
valores con el fin de estudiar cómo influyen sobre los resultados.

\textbf{Factores incontrolables:} Son aquellos parámetros o características del 
producto o proceso, que es imposible de controlar al momento de
desarrollar el experimento.

\textbf{Variabilidad natural:} es la variación entre las unidades experimentales, 
que el experimentador no puede controlar ni eliminar.

\textbf{Variable dependiente:} es la variable que se desea examinar o estudiar en 
un experimento. (Variable Respuesta).

\textbf{Hipótesis:}

Es una suposición o conjetura que se plantea el investigador de una realidad
desconocida.

Es el supuesto que se hace sobre el valor de un parámetro (constante que caracteriza
a una población) el cual puede ser validado mediante una prueba estadística.

\newpage
\chapter*{Diseños unifactoriales}
En el análisis de los resultados de los experimentos se pueden observar diferentes
aplicaciones de los Diseños Experimentales. Hay experimentos muy útiles en los cuales existe
un sólo factor de interés; el cual se analiza por medio de la comparación de dos condiciones
que intervienen en el Experimento (a menudo llamadas tratamientos o niveles del factor); a
este tipo de experimentos se le denomina Experimentos de Comparación Simple,
estudiados en los cursos de Estadística básica. El análisis de los datos de este tipo de
Experimentos resulta ser sencillo, ya que se utilizan técnicas de la Inferencia Estadística,
llamada Prueba de Hipótesis (o pruebas de significancia) que son las que ayudan al
experimentador a comparar estas condiciones.

Si en el tipo de Diseño Experimental planteado anteriormente se requiere más de dos
niveles del factor que se analiza, éstos son considerados como "Diseños Unifactoriales".
Teniendo en cuenta que para el análisis de éstos se utiliza el Análisis de Varianza, ya que se
requiere probar la igualdad de varias medias.

En los experimentos de los Diseños Unifactoriales, el número de observaciones
recolectadas en cada tratamiento pueden ser iguales o diferentes. Cuando el número de
observaciones sea diferente se dice que el Diseño está Desequilibrado o Desbalanceado; en
caso contrario el Diseño está Equilibrado o Balanceado.

En este documento se hará uso del caso en el que el Diseño está Equilibrado o Balanceado y se utilizará un modelo de efectos fijos (en este tipo
de Modelos se desea probar hipótesis en relación a las medias de los tratamientos y las
conclusiones sólo se aplicarán a los niveles del factor considerados en el análisis, las
conclusiones no pueden extenderse a tratamientos similares que no se consideraron.) y no se profundizara mas teóricamente ya que el fin de este documento es la aplicación de código del software estadistico R para la aplicación de dicho modelo.

\textbf{Ejemplo 1.}

El Ministerio de Educación esta interesado en implementar tres programas de estudio; con el
objetivo de medir la habilidad de lectura en los alumnos. Para ello, se eligen alumnos del sexto
grado de un Colegio de San Salvador, de los cuales fueron asignados al azar 27 alumnos, a
cada uno de los tres grupos. Se utilizó un programa diferente en cada grupo, se llevó a cabo un
examen al inicio y al final de la implementación de los programas, los valores obtenidos
representan la diferencia que hay entre la nota del examen que se hizo al inicio y al final de la
implementación del programa, obteniéndose los siguientes datos, en base 100:
\begin{table}[htb]
\centering
\begin{tabular}{||c|c|c|c|c|c|c|c|c|c||}
\hline
\hline
Tratamiento  & \multicolumn{9}{c||}{Datos} \\
\hline
Programa 1 &20&18&18&23&22&17&15&13&21 \\
\hline
Programa 2 &15&20&13&12&16&17&21&15&13 \\
\hline
Programa 3 &12&15&18&20&18&17&10&24&16 \\
\hline
\hline

\end{tabular}
\caption{Diferencias entre nota de inicio y nota final}
\end{table}

\textbf{Solución.}
Antes de realizar los cálculos matemáticos, se definirá la variable de estudio y las hipótesis
que se desean probar.

\textbf{Variable de estudio:} Habilidad de Lectura.

\textbf{Hipótesis:}

$H_{0}$ : $\mu_1 = \mu_2 = \mu_3$ (No existe diferencia entre los grupos)

$H_{1}$ : $\mu_1 \not= \mu_2 \not= \mu_3$ (Existe diferencia entre los grupos)

\textbf{El significado verbal de las hipótesis es:}

\textbf{$H_{0}$} : Con la implementación de los tres programas de estudio, no existe diferencia
significativa en la habilidad de lectura entre los grupos de alumnos del sexto grado.

\textbf{$H_{1}$} : Con la implementación de los tres programas de estudio, existe diferencia significativa en la habilidad de lectura entre los grupos de alumnos del sexto grado.

\begin{Schunk}
\begin{Sinput}
> library(xtable)
> #Leemos la hoja de datos
> datos<-read.csv("Ej1HL.csv", header = TRUE, sep = ";")
> #Analysis of Variance(aov) para un factor
> unifact<- aov(HL~Programa, data = datos)
> Resultado<-summary(unifact)
> #su equivalente
> Resultado2<- anova(unifact)
> #genarando el codigo latex para la tabla
> # de ANOVA
> Resultado3<- xtable(Resultado2) 
\end{Sinput}
\end{Schunk}

\begin{table}[ht]
\centering
\begin{tabular}{lrrrrr}
  \hline
 & Df & Sum Sq & Mean Sq & F value & Pr($>$F) \\ 
  \hline
Programa    & 2 & 36.22 & 18.11 & 1.44 & 0.2566 \\ 
  Residuals   & 24 & 301.78 & 12.57 &  &  \\ 
   \hline
\end{tabular}
\caption{Anova de un factor}
\end{table}

\textbf{Conclusión}

Si se observa la columna Pr(>F) ésta muestra el P-valor generado por el ANOVA y se interpreta de la siguiente manera, como el nivel de significancia de la prueba F con 2 grados de libertad en el numerador y 24 en el denominador es de del 5 por ciento y el P-valor es mayor a este, no hay sufciente evidencia estadistica para rechazar la hipostesis nula y puede concluirse que con la implementación de los tres programas de estudio, no existe diferencia
significativa en la habilidad de lectura entre los grupos de alumnos del sexto grado.
\newpage
Si al efectuar el Análisis de Varianza para un Modelo de Efectos Fijos, la hipótesis nula es
rechazada. Se llega a la conclusión que existe diferencia entre las medias o que hay diferencia
entre los tratamientos. En muchas situaciones en la industria, este resultado es de poco
interés; ya que no se especifica exactamente cuales tratamientos son diferentes y el
experimentador espera hallar diferencias, y está más interesado en investigar que tratamientos
difieren entre si, o dicho de otra manera, en investigar contrastes entre los tratamientos.

Cuando se da esta situación puede ser útil realizar comparaciones adicionales entre grupos
de medias de los tratamientos. Las comparaciones entre medias de tratamientos se realizan en
términos de los totales de tratamientos $y_{i.}$ o de los promedios de tratamientos $\bar{y_{i.}}$. Los procedimientos para efectuar esta comparación se conocen como Métodos de Comparación
Múltiple o Pruebas a Posteriori.

Esto quiere decir que si no existe diferencia significativa entre la media de los tratamientos no tiene sentido aplicar pruebas posteriori, sin embargo por efectos de la utilización del código en R, se aplicará la prueba de tukey ("TukeyHSD" en R) cuya hipótesis nula es: "No esxiste difencia significativa entre las comparaciones".

\textbf{Estimacion de las medias de los tratamientos:}
\begin{Schunk}
\begin{Sinput}
> library(ggplot2)
> library(gplots)
> boxplot(datos$HL~datos$Programa,
+         ylab = "Habilidad de lectura",
+         xlab = "Tratamientos",
+         col=rainbow(5, alpha = .3))
> plotmeans(datos$HL~datos$Programa,
+           ylab = "Habilidad de lectura",
+           xlab = "Tratamientos")
\end{Sinput}
\end{Schunk}
\newpage
\textbf{Evaluacion de las diferencias entre los tratamientos:}

Prueba de Tukey: 
\begin{Schunk}
\begin{Sinput}
> data.Prueba<-TukeyHSD(unifact, ordered = TRUE)
> data.Prueba.result<- data.frame(data.Prueba$Programa)
> Tabla<- xtable(data.Prueba.result)
\end{Sinput}
\end{Schunk}

\begin{table}[ht]
\centering
\begin{tabular}{rrrrr}
  \hline
 & diff & lwr & upr & p.adj \\ 
  \hline
Programa3-Programa2 & 0.89 & -3.29 & 5.06 & 0.86 \\ 
  Programa1-Programa2 & 2.78 & -1.40 & 6.95 & 0.24 \\ 
  Programa1-Programa3 & 1.89 & -2.29 & 6.06 & 0.51 \\ 
   \hline
\end{tabular}
\caption{TukeyHSD}
\end{table}

Puede observarse que el P-valor en la columna P.adj es superior al 5 porciento por lo que se concluye que no existe diferencia significativa entre las medias que se han comparado, el resultado es obvio dado que en el analisis de varianza se conluyo que no existia diferencia significativa entre las medias de los tratamientos.

\textbf{Ejemplo 2.}

Se supone que la cantidad de carbón usada en la producción de acero tiene un efecto en su
resistencia a la tensión. En la tabla se muestran los valores de la resistencia a la tensión del
acero para cada uno de los 4 diferentes porcentajes de carbón. Con estos datos efectúe el
análisis apropiado e interprete sus resultados.

\textbf{Variable de estudio:} Resistencia a la tensión.

\textbf{Hipótesis:}

$H_{0}$ : $\mu_1 = \mu_2 = \mu_3 = \mu_4$ (No existe diferencia entre los grupos)

$H_{1}$ : $\mu_1 \not= \mu_2 \not= \mu_3 \not= \mu_4$ (Existe diferencia entre los grupos)

\textbf{El significado verbal de las hipótesis es:}

\textbf{$H_{0}$} : La cantidad de carbón usada en la producción de acero no tiene efecto significativo en la resistencia a la tensión

\textbf{$H_{1}$} : La cantidad de carbón usada en la producción de acero tiene efecto significativo en la resistencia a la tensión.

\begin{table}[htb]
\centering
\begin{tabular}{||c|c|c|c|c||}
\hline
\hline
\% de Carbon  & \multicolumn{4}{c||}{Observaciones} \\
\hline
0.10 &23&28&28&30 \\
\hline
0.20 &31&29&36&38 \\
\hline
0.30 &36&40&42&44 \\
\hline
0.40 &48&45&40&40 \\
\hline
\hline

\end{tabular}
\caption{Resistencia a la Tensión}
\end{table}

\begin{Schunk}
\begin{Sinput}
> FCarbon<-c("0.10%","0.10%","0.10%",
+            "0.10%","0.20%","0.20%",
+            "0.20%","0.20%","0.30%",
+            "0.30%","0.30%","0.30%",
+            "0.40%","0.40%","0.40%",
+            "0.40%")
> Nivel<-factor(FCarbon)
> Datos<-c(23,28,28,30,31,29,36,38,36,40,42,44,48,45,40,40)
> Unifact<-aov(Datos~Nivel)
> Result<-summary(Unifact)
> Tabla <- xtable(Result)
\end{Sinput}
\end{Schunk}

\begin{table}[ht]
\centering
\begin{tabular}{lrrrrr}
  \hline
 & Df & Sum Sq & Mean Sq & F value & Pr($>$F) \\ 
  \hline
Nivel       & 3 & 622.25 & 207.42 & 15.41 & 0.0002 \\ 
  Residuals   & 12 & 161.50 & 13.46 &  &  \\ 
   \hline
\end{tabular}
\caption{Anova de un factor}
\end{table}
\newpage
\textbf{Conclusión.}

Se concluye que la cantidad de carbón usada en la producción de acero tiene efectos
significativos en la resistencia a la tensión.
Como se ha rechazado $H_0$, existe diferencia entre las medias de los tratamientos, pero no
se especifica entre que medias de tratamientos existen las diferencias.
Se podría estar interesado en querer saber entre que medias de tratamientos existe
diferencia, para ello se utilizará el método de Tukey y LSD para contestar esta inquietud.

\textbf{Estimacion de las medias de los tratamientos:}
\begin{Schunk}
\begin{Sinput}
> library(ggplot2)
> library(gplots)
> boxplot(Datos~FCarbon,
+         ylab = "Resistencia a la tensión",
+         xlab = "Porcentaje de Carbón",
+         col=rainbow(5, alpha = .3))
> plotmeans(Datos~FCarbon,
+           ylab = "Resistencia a la tensión",
+           xlab = "Porcentaje de Carbón")
\end{Sinput}
\end{Schunk}

\textbf{Evaluacion de las diferencias entre los tratamientos:}

Prueba LSD:
\begin{Schunk}
\begin{Sinput}
> library(agricolae)
> data.Prueba1<- LSD.test(Unifact, "Nivel" )
> data.Prueba1.result1<- data.frame(data.Prueba1$statistics)
> data.Prueba1.result2<- data.frame(data.Prueba1$means)
> data.Prueba1.result3<- data.frame(data.Prueba1$groups)
> tabla1 <- xtable(data.Prueba1.result1)
> tabla2 <- xtable(data.Prueba1.result2)
> tabla3 <- xtable(data.Prueba1.result3)
\end{Sinput}
\end{Schunk}

\begin{table}[ht]
\centering
\begin{tabular}{rrrrr}
  \hline
 & Mean & CV & MSerror & LSD \\ 
  \hline
  & 36.12 & 10.16 & 13.46 & 5.65 \\ 
   \hline
\end{tabular}
\caption{Calculo del LSD}
\end{table}

\begin{table}[ht]
\centering
\begin{tabular}{rrrrrrrr}
  \hline
Porcentaje de carbón  & Means & std & r & LCL & UCL & Min & Max \\ 
  \hline
0.10\% & 27.25 & 2.99 &   4 & 23.25 & 31.25 & 23.00 & 30.00 \\ 
  0.20\% & 33.50 & 4.20 &   4 & 29.50 & 37.50 & 29.00 & 38.00 \\ 
  0.30\% & 40.50 & 3.42 &   4 & 36.50 & 44.50 & 36.00 & 44.00 \\ 
  0.40\% & 43.25 & 3.95 &   4 & 39.25 & 47.25 & 40.00 & 48.00 \\ 
   \hline
\end{tabular}
\caption{Estadisticos de los Tratamientos}
\end{table}

\begin{table}[ht]
\centering
\begin{tabular}{rlrl}
  \hline
 & trt & means & M \\ 
  \hline
  1 & 0.40\% & 43.25 & a \\ 
  2 & 0.30\% & 40.50 & a \\ 
  3 & 0.20\% & 33.50 & b \\ 
  4 & 0.10\% & 27.25 & c \\ 
   \hline
\end{tabular}
\caption{Comparaciones}
\end{table}

Las conclusiones principales se toman apartir de el cuadro 8, de manera que si se observa la columna "M" se deben hacer las combinaciones de las letras y éstas diferiran en media significativamente si en las combinaciones las letras no se repiten, es decir, la media en la resistencia a la tensión según en el porcentaje de carbón para los grupos 1 y 2 (30\% y 40\%) no difiere significativamente, sin embargo para las demas combinaciones 1 y 3, 1 y 4, 2 y 3, 2 y 4, 3 y 4, si difieren significativamente en la media de la resistencia, Para reforzar esta conlclusión se realizara la prueba de tukey.
\newpage
Prueba de Tukey: 
\begin{Schunk}
\begin{Sinput}
> data.Prueba<-TukeyHSD(Unifact, ordered = TRUE)
> data.Prueba.result<- data.frame(data.Prueba$Nivel)
> Tabla<- xtable(data.Prueba.result)
\end{Sinput}
\end{Schunk}

\begin{table}[ht]
\centering
\begin{tabular}{rrrrr}
  \hline
 Diferencias  & diff & lwr & upr & p.adj \\ 
  \hline
0.20\%-0.10\% & 6.25 & -1.45 & 13.95 & 0.13 \\ 
  0.30\%-0.10\% & 13.25 & 5.55 & 20.95 & 0.00 \\ 
  0.40\%-0.10\% & 16.00 & 8.30 & 23.70 & 0.00 \\ 
  0.30\%-0.20\% & 7.00 & -0.70 & 14.70 & 0.08 \\ 
  0.40\%-0.20\% & 9.75 & 2.05 & 17.45 & 0.01 \\ 
  0.40\%-0.30\% & 2.75 & -4.95 & 10.45 & 0.72 \\ 
   \hline
\end{tabular}
\caption{TukeyHSD}
\end{table}

Y efectivamente puede observarse que en la columna P.adj la comparación entre los grupos con 40\% y 30\% de carbón no difieren significativamente en media (igual al LSD) sin embargo en los grupos 20\% y 10\% y 30\% y 20\% de carbón  dado que el P.adj es mayor a 0.05 no existe diferencia significativa en las medias de estas comparaciones y si existe diferencia en las demas combinaciones, en definitiva puede decirse que el la prueba de tukey es mas precisa que el la prueba LSD dado que esta reporto un grupo más que no difieren significativamente en media.
\newpage
\chapter*{Diseños por Bloques}
En todo Diseño de Experimento se desea que el Error Experimental sea lo más pequeño
posible, ya que refleja tanto el Error aleatorio como la variabilidad entre las unidades
experimentales o grupos de Unidades Experimentales; es decir, se trata de sustraer del Error
Experimental la variabilidad producida por las unidades experimentales o grupos de Unidades
Experimentales.

Para lograr lo anterior, las unidades experimentales se deben distribuir aleatoriamente
en grupos o bloques, de tal manera que dentro de un bloque sean relativamente homogéneas y
que el número de ellas dentro de un bloque sea igual al número de tratamientos por investigar.
A esta forma de organizar y distribuir las unidades Experimentales se le denomina \textbf{Diseño por Bloques.}

El principio de bloqueo se basa en que las unidades experimentales dentro de cada
bloque o grupo deben ser parecidas entre si (que exista homogeneidad dentro del bloque) y
que los bloques debieran ser diferentes entre si (que exista heterogeneidad entre los bloques);
y además que las unidades experimentales se deben distribuir aleatoriamente en los grupos o
bloques de tal manera que dentro de un bloque sean relativamente homogéneas; lo que
significa que el bloqueo o agrupamiento del material experimental debe hacerse de tal forma
que las unidades experimentales dentro de un bloque sean tan homogéneas como sea posible y
que las diferencias entre las unidades experimentales sean explicadas, en mayor proporción,
por las diferencias entre los bloques. Esta técnica es una generalización de la prueba de
comparación de medias de muestras apareadas, donde cada pareja es un bloque. Para la
validez del análisis, se supone que no existe interacción entre los bloques y los tratamientos.
En experimentos de campo, cuando no se tenga información segura sobre la
uniformidad del terreno, siempre es preferible introducir bloques, que servirán para separar la
variabilidad debido a dicha heterogeneidad.

Los Diseños que cumplen estas características y que son objeto de estudio, se
clasifican en:
\begin{enumerate}
  \item Diseños Aleatorizados por Bloques Completos.
  \item Diseños Aleatorizados por Bloques Incompletos Balanceados.
\end{enumerate}
En este documento se hara referencia unicamente a los Diseños Aleatorizados por Bloques Completos este Diseño se da cuando cada bloque tiene tantas unidades Experimentales como
Tratamientos, y todos los tratamientos son asignados al azar dentro de cada bloque.
En este tipo de Diseño la palabra ”Completo” significa que todos los tratamientos son
probados en cada bloque, y los bloques forman una unidad experimental más homogénea con
la cual se comparan los tratamientos, y ”Aleatorizado” porque dentro de cada bloque los
tratamientos son asignados de forma aleatoria a las unidades Experimentales; es decir, que
todas las Unidades Experimentales de un mismo bloque tienen la misma probabilidad de recibir
cualquiera de los tratamientos. Esta estrategia de Diseño mejora efectivamente la precisión en
las comparaciones al eliminar la variabilidad en los tratamientos.
\newpage
\textbf{Ejemplo 1.}

Se probaran 5 raciones respecto a sus diferencias en el engorde de novillos. Se dispone de 20
novillos para el experimento, que se distribuyen en 4 bloques (5 novillos por bloque) con base
a sus pesos, al iniciar la prueba de engorde. Los 5 tratamientos (raciones) se asignaron al azar
dentro de cada bloque. Los novillos más pesados se agruparon en un bloque, en otro se
agruparon los 5 siguientes más pesados y así sucesivamente. Se obtuvieron los siguientes
datos:

\begin{table}[htb]
\centering
\begin{tabular}{||c|c|c|c|c||}
\hline
\hline
\multirow{2}{*}{Raciones}  & \multicolumn{4}{|c||}{Bloques} \\
\cline{2-5}
     &1&2&3&4 \\
\hline
Ración 1 &0.9&1.4&1.4&2.3 \\
\hline
Ración 2 &3.6&3.2&4.5&4.1 \\
\hline
Ración 3 &0.5&0.9&0.5&0.9 \\
\hline
Ración 4 &3.6&3.6&3.2&3.6 \\
\hline
Ración 5 &1.8&1.8&0.9&1.4 \\
\hline
\hline

\end{tabular}
\caption{Peso de los novillos}
\end{table}

\textbf{Solución}

Antes de realizar los cálculos matemáticos, se definirán las hipótesis que se desean probar.

$H_0 : \mu_1 = \mu_2 = \mu_3 = \mu_4 = \mu_5$ (No existe diferencia entre las Raciones)
$H_1 : \mu_1 \not= \mu_2 \not= \mu_3 \not= \mu_4 \not= \mu_5$ (Existe diferencia entre las Raciones)

\textbf{Variable Respuesta:} Peso de los Novillos.

\textbf{El significado verbal es:}

$H_0$ : La cantidad de ración no influye en el engorde de los novillos.
$H_1$ : La cantidad de ración influye en el engorde de los novillos


\begin{Schunk}
\begin{Sinput}
> library(xtable)
> #Leemos la hoja de datos
> datos<-read.csv("Ej1_B.csv", header = TRUE, sep = ";")
> #Analysis of Variance(aov) para un factor
> DBloques<- aov(datos$Observaciones ~ datos$Raciones + datos$Bloques)
> Resultado<-summary(DBloques)
> #su equivalente
> Resultado2<- anova(DBloques)
> #genarando el codigo latex para la tabla
> # de ANOVA
> Resultado3<- xtable(Resultado2) 
\end{Sinput}
\end{Schunk}

\begin{table}[ht]
\centering
\begin{tabular}{lrrrrr}
  \hline
 & Df & Sum Sq & Mean Sq & F value & Pr($>$F) \\ 
  \hline
Raciones & 4 & 30.71 & 7.68 & 39.11 & 0.0000 \\ 
Bloques(novillos) & 3 & 0.46 & 0.15 & 0.78 & 0.5257 \\ 
  Residuals & 12 & 2.36 & 0.20 &  &  \\ 
   \hline
\end{tabular}
\caption{Análisis de Varianza}
\end{table}

\textbf{Conclusión}

Las cinco raciones no son igualmente efectivas en el engorde de novillos o la cantidad de
ración influye significativamente en el engorde de los novillos, otro punto importante que puede observarse es que los bloques (novillos) no son significativos, es decir no hay un efecto significativo de los bloques hacia el factor, o dicho de otra manera separar a los novillos en bloques no reduce significativamente la variabilidad, en palabras sencillas, dado que los bloques no son significativos, da igual utilizar un diseño unifactorial.

Como la hipotesis nula fue rechazada es necesario determinar
que medias son significativamente diferentes; para observar esas diferencias entre las medias
se utilizará la prueba LSD y tukey.

\textbf{Estimacion de las medias de los tratamientos:}
\begin{Schunk}
\begin{Sinput}
> library(ggplot2)
> library(gplots)
> boxplot(datos$Observaciones~datos$Raciones,
+         ylab = "Pesos",
+         xlab = "Raciones",
+         col=rainbow(5, alpha = .3))
> plotmeans(datos$Observaciones~datos$Raciones,
+           ylab = "Pesos",
+           xlab = "Raciones")
\end{Sinput}
\end{Schunk}

Prueba LSD:
\begin{Schunk}
\begin{Sinput}
> library(agricolae)
> data.Prueba1<- LSD.test(DBloques, "datos$Raciones")
> data.Prueba1.result1<- data.frame(data.Prueba1$statistics)
> data.Prueba1.result2<- data.frame(data.Prueba1$means)
> data.Prueba1.result3<- data.frame(data.Prueba1$groups)
> tabla1 <- xtable(data.Prueba1.result1)
> tabla2 <- xtable(data.Prueba1.result2)
> tabla3 <- xtable(data.Prueba1.result3)
\end{Sinput}
\end{Schunk}

\begin{table}[ht]
\centering
\begin{tabular}{rrrrr}
  \hline
 & Mean & CV & MSerror & LSD \\ 
  \hline
  & 2.21 & 20.10 & 0.20 & 0.68 \\ 
   \hline
\end{tabular}
\caption{Calculo del LSD}
\end{table}

\begin{table}[ht]
\centering
\begin{tabular}{rrrrrrrr}
  \hline
  Raciones  & Means & std & r & LCL & UCL & Min & Max \\ 
  \hline
  R1 & 1.50 & 0.58 &   4 & 1.02 & 1.98 & 0.90 & 2.30 \\ 
  R2 & 3.85 & 0.57 &   4 & 3.37 & 4.33 & 3.20 & 4.50 \\ 
  R3 & 0.70 & 0.23 &   4 & 0.22 & 1.18 & 0.50 & 0.90 \\ 
  R4 & 3.50 & 0.20 &   4 & 3.02 & 3.98 & 3.20 & 3.60 \\ 
  R5 & 1.48 & 0.43 &   4 & 0.99 & 1.96 & 0.90 & 1.80 \\ 
   \hline
\end{tabular}
\caption{Estadisticos}
\end{table}

\begin{table}[ht]
\centering
\begin{tabular}{rlrl}
  \hline
  & trt & means & M \\ 
  \hline
  1 & R2 & 3.85 & a \\ 
  2 & R4 & 3.50 & a \\ 
  3 & R1 & 1.50 & b \\ 
  4 & R5 & 1.48 & b \\ 
  5 & R3 & 0.70 & c \\ 
   \hline
\end{tabular}
\caption{Comparaciones}
\end{table}
\newpage
\textbf{Conclusión}

Puede observarse que las combinaciones 1 y 2, 3 y 4 no difieren significativamente en media y las combinaciones 1 y 3, 1 y 4, 1 y 5, 2 y 3, 2 y 4, 2 y 5, 3 y 5, 4 y 5, difieren significativamente en media, en el contexto las raciones que se le den al novillo si afectan en su peso.  

Prueba de Tukey:
\begin{Schunk}
\begin{Sinput}
> data.Prueba<-TukeyHSD(DBloques, ordered = TRUE)
> data.Prueba.result<- data.frame(data.Prueba$`datos$Raciones`)
> Tabla<- xtable(data.Prueba.result)
\end{Sinput}
\end{Schunk}

\begin{table}[ht]
\centering
\begin{tabular}{rrrrrr}
  \hline
      & & diff & lwr & upr & p.adj \\ 
  \hline
1&R5-R3 & 0.78 & -0.22 & 1.77 & 0.16 \\ 
2&R1-R3 & 0.80 & -0.20 & 1.80 & 0.14 \\ 
3&R4-R3 & 2.80 & 1.80 & 3.80 & 0.00 \\ 
4&R2-R3 & 3.15 & 2.15 & 4.15 & 0.00 \\ 
5&R1-R5 & 0.02 & -0.97 & 1.02 & 1.00 \\ 
6&R4-R5 & 2.03 & 1.03 & 3.02 & 0.00 \\ 
7&R2-R5 & 2.38 & 1.38 & 3.37 & 0.00 \\ 
8&R4-R1 & 2.00 & 1.00 & 3.00 & 0.00 \\ 
9&R2-R1 & 2.35 & 1.35 & 3.35 & 0.00 \\ 
10&R2-R4 & 0.35 & -0.65 & 1.35 & 0.79 \\ 
   \hline
\end{tabular}
\caption{TukeyHSD}
\end{table}
\newpage
\textbf{Conclusión}

Puede observarse que en las combinaciones 1,2,5,10 no existe diferencia significativa en la media del peso de los novillos.

\newpage

\textbf{Ejemplo 2.}

Un ingeniero de control de calidad para manufacturas de componentes electrónicos,
intuitivamente siente que hay mucha variabilidad entre los 36 hornos usados por la compañía
manufacturera para la cual trabaja, en las pruebas de duración de los diversos componentes.
Para determinar si tiene razón, escoge un sólo tipo de componente y obtiene los siguientes
datos para las 2 temperaturas (T) normalmente usadas en las pruebas de duración de estas
unidades; T1 = 550ºF y T2 = 600ºF. El componente se pone a funcionar en un horno hasta que
falla. En este experimento se utilizaron 3 hornos (H) seleccionados aleatoriamente y se midió
los minutos de duración del componente electrónico. Con un 5\% de significancia, pruebe si el
ingeniero tiene razón en su creencia de la diferencia entre los 36 hornos. Los datos obtenidos
son los siguiente:

\begin{table}[htb]
\centering
\begin{tabular}{||c|c|c||}
\hline
\hline
\multirow{2}{*}{Hornos}  & \multicolumn{2}{|c||}{Temperatura} \\
\cline{2-3}
         &550ºF&600ºF \\
\hline
$H_1$       &246&180 \\
\hline
$H_2$       &191&144 \\
\hline
$H_3$       &187&134 \\
\hline
\hline

\end{tabular}
\caption{Temperaturas}
\end{table}

\textbf{Solución}

Antes de realizar los cálculos matemáticos, se definirán las hipótesis que se desean probar; ya
que hay dieciséis hornos y solo se toman tres aleatoriamente es un Modelo de Efectos
aleatorios.

$H_0 : \sigma^{2}_t = 0$ (No existe variabilidad entre los Hornos)

$H_1 : \sigma^{2}_t > 0$ (Existe variabilidad entre los Hornos)

\textbf{Variable Respuesta:} Duración del componente electrónico.

\textbf{El significado verbal es:}

$H_0$ : La variabilidad entre los Hornos no influyen significativamente en la duración del
componente electrónico probado.

$H_1$ : La variabilidad entre los Hornos influyen significativamente en la duración del componente
electrónico probado.

\begin{Schunk}
\begin{Sinput}
> library(xtable)
> #Leemos la hoja de datos
> datos<-read.csv("Ej2_B.csv", header = TRUE, sep = ";")
> #Analysis of Variance(aov) para un factor
> DBloques<- aov(datos$Obs ~ datos$Hornos + datos$Temperaturas)
> Resultado<-summary(DBloques)
> #su equivalente
> Resultado2<- anova(DBloques)
> #genarando el codigo latex para la tabla
> # de ANOVA
> Resultado3<- xtable(Resultado2) 
\end{Sinput}
\end{Schunk}

\begin{table}[ht]
\centering
\begin{tabular}{lrrrrr}
  \hline
 & Df & Sum Sq & Mean Sq & F value & Pr($>$F) \\ 
  \hline
Hornos & 2 & 3250.33 & 1625.17 & 34.46 & 0.0282 \\ 
Temperaturas & 1 & 4592.67 & 4592.67 & 97.37 & 0.0101 \\ 
  Residuals & 2 & 94.33 & 47.17 &  &  \\ 
   \hline
\end{tabular}
\caption{Análisis de Varianza}
\end{table}

\textbf{Conclusión}

Puede observarse de la tabla de análsis e varianza que tanto bloques (temperaturas) y hornos son significativos por lo tanto, los tipos de hornos influyen significativamente en la duración del componente electrónico probado y separar los hornos en bloques por temperaturas ayuda a reducir la variabilidad del factor, el cual es el proposito del bloque.

Como en el ejercicio 2, fue rechazada la hipótesis nula. Supongamos que se desea
saber cuales son las parejas de medias que son diferentes; para ello se utilizará el Método de la
Mínima Diferencia Significativa (LSD) y tukey.

\textbf{Estimacion de las medias de los tratamientos:}
\begin{Schunk}
\begin{Sinput}
> library(ggplot2)
> library(gplots)
> boxplot(datos$Obs~datos$Hornos,
+         ylab = "Temperaturas",
+         xlab = "Hornos",
+         col=rainbow(5, alpha = .3))
> plotmeans(datos$Obs~datos$Hornos,
+           ylab = "Temperaturas",
+           xlab = "Hornos")
\end{Sinput}
\end{Schunk}

Prueba LSD:
\begin{Schunk}
\begin{Sinput}
> library(agricolae)
> data.Prueba1<- LSD.test(DBloques, "datos$Hornos")
> data.Prueba1.result1<- data.frame(data.Prueba1$statistics)
> data.Prueba1.result2<- data.frame(data.Prueba1$means)
> data.Prueba1.result3<- data.frame(data.Prueba1$groups)
> tabla1 <- xtable(data.Prueba1.result1)
> tabla2 <- xtable(data.Prueba1.result2)
> tabla3 <- xtable(data.Prueba1.result3)
\end{Sinput}
\end{Schunk}
\newpage
\begin{table}[ht]
\centering
\begin{tabular}{rrrrr}
  \hline
 & Mean & CV & MSerror & LSD \\ 
  \hline
  & 180.33 & 3.81 & 47.17 & 29.55 \\ 
   \hline
\end{tabular}
\caption{Calculo del LSD}
\end{table}

\begin{table}[ht]
\centering
\begin{tabular}{rrrrrrrr}
  \hline
 & datos.Obs & std & r & LCL & UCL & Min & Max \\ 
  \hline
H1 & 213.00 & 46.67 &   2 & 192.11 & 233.89 & 180.00 & 246.00 \\ 
  H2 & 167.50 & 33.23 &   2 & 146.61 & 188.39 & 144.00 & 191.00 \\ 
  H3 & 160.50 & 37.48 &   2 & 139.61 & 181.39 & 134.00 & 187.00 \\ 
   \hline
\end{tabular}
\caption{Estadísticos}
\end{table}

\begin{table}[ht]
\centering
\begin{tabular}{rlrl}
  \hline
 & trt & means & M \\ 
  \hline
1 & H1 & 213.00 & a \\ 
  2 & H2 & 167.50 & b \\ 
  3 & H3 & 160.50 & b \\ 
   \hline
\end{tabular}
\caption{Comparaciones}
\end{table}

\textbf{Conclusión}

Se observa que la pareja de medias que no difieren significativamente son la media dos
y la media tres; por lo tanto, no existe diferencia significativa entre el Horno dos y tres, las parejas de medias uno y dos, uno y tres al hacer las combinaciones las letras se no se repiten; por lo tanto, las dos
parejas de medias difieren significtivamente; es decir que existe diferencia significativa
entre el Horno uno con los Hornos dos y tres.

Prueba de Tukey:
\begin{Schunk}
\begin{Sinput}
> data.Prueba<-TukeyHSD(DBloques, ordered = TRUE)
> data.Prueba.result<- data.frame(data.Prueba$`datos$Hornos`)
> Tabla<- xtable(data.Prueba.result)
\end{Sinput}
\end{Schunk}

\begin{table}[ht]
\centering
\begin{tabular}{rrrrrr}
  \hline
 & diff & lwr & upr & p.adj \\ 
  \hline
  1&H2-H3 & 7.00 & -33.46 & 47.46 & 0.64 \\ 
  2&H1-H3 & 52.50 & 12.04 & 92.96 & 0.03 \\ 
  3&H1-H2 & 45.50 & 5.04 & 85.96 & 0.04 \\ 
   \hline
\end{tabular}
\caption{TukeyHSD}
\end{table}
\newpage
\textbf{Conclusión}

Puede notarse que el P.adj para la comparacion 1 es mayor a 0.05 por lo que no existe diferencia significativa entre las medias del horno 2 y 3.

\textbf{Con respecto a las Temperaturas:}

Pudo observarse que adiferencia del ejemplo 1, en el ejemplo dos los bloques si son significativos por lo tanto puede decirse que éstos tiene un efecto significativo en los componenetes electronicos, para saber cual es su efecto se realiza el siguiente procedimiento:

\begin{Schunk}
\begin{Sinput}
> model.tables(DBloques)
\end{Sinput}
\begin{Soutput}
Tables of effects

 datos$Hornos 
datos$Hornos
    H1     H2     H3 
 32.67 -12.83 -19.83 

 datos$Temperaturas 
datos$Temperaturas
     T1      T2 
 27.667 -27.667 
\end{Soutput}
\end{Schunk}
Para los hornos en H1 aumentara la temperatura si se colocan los componentes en el horno 1. y disminuirá en -12.83ºF y -19.83ºF si se coloca el componente en el Horno 2 y 3 respectivamente. 

El 27.667 para T1 significa que la tempreratura aumentara en 27.667ºF si se colocan los componentes en el Bloque 1 y disminuira -27.667ºF si se coloca en el Bloque 2.

\chapter*{Diseños Factoriales}
En muchas situaciones experimentales resulta de interés estudiar los efectos producidos
por dos o más factores simultáneamente; esto se logra con la ayuda de los Diseños Factoriales.
En general los Diseños Factoriales producen experimentos más eficientes, ya que cada
observación proporciona información sobre todos los factores, y es posible ver las respuestas
de un factor en diferentes niveles de otro factor en el mismo experimento. Por lo tanto, se
entiende por Diseño Factorial a aquel diseño en el cual se pueden estudiar los efectos de dos
o más factores de variación a la vez; es decir, que se puede investigar todas las posibles
combinaciones de los niveles de los factores en cada ensayo completo o réplica del
experimento.
Cada uno de los factores en estudio varían en su aplicación, a esta variación se le llama
Niveles del Factor. Las combinaciones de los niveles de cada factor, forman los respectivos
tratamientos.
En un diseño factorial, los factores en estudio se representan por letras mayúsculas
(A,B,C,……) y los niveles de cada uno por sus respectivas letras minúsculas (a,b,c,……). Los
cuales pueden tomar valores de 2,3,4, ……
Existen experimentos factoriales Balanceados y Desbalanceados; diremos que es
balanceado cuando el número de réplicas es igual para cada uno de los tratamientos usados en
el experimento; en caso contrario es Desbalanceado; también se puede dar el caso en que sólo
exista una sola réplica para cada tratamiento. Los Diseños factoriales se pueden combinar con
los Diseños Completamente al Azar (Unifactoriales), o con el Diseño de Bloques Aleatorios, etc.,
dependiendo de la naturaleza del experimento.

Entre las Ventajas de usar un diseño Factorial, se pueden mencionar las siguientes:
\begin{enumerate}
\item Ahorro y economía del recurso experimental; ya que cada unidad experimental provee
información acerca de dos o más factores, lo que no sucede cuando se realiza con una serie
de experimentos simples.

\item Da información respecto a las interacciones entre los diversos factores en estudio.

\item Permite realizar estimaciones de las interacciones de los factores, además de los efectos
simples.

\item Permite estimar los efectos de un factor en diversos niveles de los otros factores,
produciendo conclusiones que son válidas sobre toda la extensión de las condiciones
experimentales.
\end{enumerate}

La única desventaja es que si el número de niveles de algunos de los factores o el número
de factores es demasiado grande, entonces el número de todas las combinaciones posibles de
tratamientos de factores llega a ser un número grande, en consecuencia la variabilidad en el
experimento podría ser grande. Estas dos situaciones, pueden hacer difícil detectar los efectos
significativos en el experimento.
Se entiende por efecto de un factor al cambio en la respuesta media ocasionada por un
cambio en el nivel de ese factor.

En los diseños factoriales existen tres efectos, los cuales son:
\begin{enumerate}
\item \textbf{Efecto Simple:} son comparaciones entre los niveles de un factor a un sólo nivel del otro
factor.
\item \textbf{Efecto Principal:} son comparaciones entre los niveles de un factor promediados para
todos los niveles del otro factor.
\item \textbf{Efecto de Interacción:} Miden las diferencias entre los efectos simples de un factor a
diferentes niveles de otro factor; es decir, la diferencia en la respuesta entre los niveles de
un factor no es la misma en todos los niveles de los otros factores.
\end{enumerate}
\newpage
\section*{Diseño Bifactorial}
El diseño factorial más simple o sencillo es aquel que involucra en su estudio sólo dos
factores o conjunto de tratamientos; es decir, que sólo se está interesado en los efectos que
producen estos dos factores. A este tipo de diseño se le llama Bifactorial.
Si A y B son los factores que se van ha estudiar en un diseño factorial, el factor A tendrá
"a niveles" y el factor B tendrá "b niveles", entonces cada repetición o réplica del experimento
contiene todas las "ab" combinaciones de los tratamientos y en general hay "n" repeticiones, es
necesario tener al menos dos réplicas (n=2), para poder obtener la suma de cuadrados del
error.

\textbf{Ejemplo 1.}

Se llevó a cabo un estudio del efecto de la temperatura sobre el porcentaje de encogimiento de
telas teñidas, con dos réplicas para cada uno de cuatro tipos de tela en un diseño totalmente
aleatorizado. Los datos son el porcentaje de encogimiento de dos réplicas de tela secadas a
cuatro temperaturas; los cuales se muestran a continuación.

\begin{table}[htb]
\centering
\begin{tabular}{||c|c|c|c|c||}
\hline
\hline
\multirow{2}{*}{Factor A (Telas)} & \multicolumn{4}{c||}{Factor B (Temperatura)} \\
\cline{2-5}
                   &210ºF&215ºF&220ºF&225ºF \\
\hline
\multirow{2}{*}{1} &1.8&2.0&4.6&7.5 \\
                   &2.1&2.1&5.0&7.9 \\
\hline
\multirow{2}{*}{2} &2.2&4.2&5.4&9.8 \\
                   &2.4&4.0&5.6&9.2 \\
\hline
\multirow{2}{*}{3} &2.8&4.4&8.7&13.2 \\
                   &3.2&4.8&8.4&13.0 \\
\hline
\multirow{2}{*}{4} &3.2&3.3&5.7&10.9 \\
                   &3.6&3.5&5.8&11.1 \\
\hline
\hline

\end{tabular}
\caption{Porcentaje de encogimiento de la tela}
\end{table}
\newpage

\textbf{Solución}

En este ejemplo el análisis se hará como un Modelo de Efectos Fijos; ya que el
investigador define con su propio criterio las temperaturas y los tipos de telas que va ha utilizar para llevar a cabo este experimento.

\textbf{Variable Respuesta:} Porcentaje de encogimiento de la tela teñida.

\textbf{Planteamiento de las Hipótesis a probar:}

Con el objetivo de ejemplificar el Análisis de Varianza de este tipo de Modelo, se plantearan
las tres hipótesis en forma Estadística que se desean probar.
\begin{enumerate}
\item $H_0 : \uptau_1 = \uptau_2 = \uptau_3 = \uptau_4 = 0$

      $H_1$ : Cuando menos un $\uptau_i \not= 0$

\item $H_0 : \beta_1 = \beta_2 = \beta_3 = \beta_4 = 0$

      $H_1$ : Cuando menos un $\beta_j \not= 0$

\item $H_0 : (\uptau\beta)_{ij} = 0$, para todo ij.

      $H_1$ : Cuando menos un $(\uptau\beta)_{ij} \not= 0$
\end{enumerate}

\textbf{Forma verbal de las Hipótesis}

\begin{enumerate}

\item $H_0$ : El tipo de tela no influye en el porcentaje de encogimiento de la tela teñida.

$H_1$ : El tipo de tela influye en el porcentaje de encogimiento de la tela teñida .

\item $H_0$ : Los niveles de temperatura no influyen en el porcentaje de encogimiento de
la tela teñida.

$H_1$ : Los niveles de temperatura influyen en el porcentaje de encogimiento de la
tela teñida.

\item $H_0$ : La combinación del tipo de tela teñida y la temperatura no influye significativamente en el porcentaje de encogimiento de la tela.

$H_1$ : La combinación del tipo de tela teñida y la temperatura influye significativamente en
el porcentaje de encogimiento de la tela.
\end{enumerate}

\begin{Schunk}
\begin{Sinput}
> library(xtable)
> #Leemos la hoja de datos
> datos<-read.csv("Ej1_F.csv", header = TRUE, sep = ";")
> #Analysis of Variance(aov) para 2 factores
> D2Fact<- aov(datos$obs~ datos$Tela*datos$Temperatura)
> Resultado<-summary(D2Fact)
> #su equivalente
> Resultado2<- anova(D2Fact)
> #genarando el codigo latex para la tabla
> # de ANOVA
> Resultado3<- xtable(Resultado2) 
\end{Sinput}
\end{Schunk}

\begin{table}[ht]
\centering
\begin{tabular}{lrrrrr}
  \hline
 & Df & Sum Sq & Mean Sq & F value & Pr($>$F) \\ 
  \hline
Tela & 3 & 41.88 & 13.96 & 279.18 & 0.0000 \\ 
Temperatura & 3 & 283.94 & 94.65 & 1892.91 & 0.0000 \\ 
Tela:Temperatura & 9 & 15.86 & 1.76 & 35.24 & 0.0000 \\ 
Residuals & 16 & 0.80 & 0.05 &  &  \\ 
   \hline
\end{tabular}
\caption{Análisis de Varianza}
\end{table}

\textbf{Conlusiones}

\begin{enumerate}

\item {Respecto a la hipótesis 1. (Factor A(Tipo de Tela))}

Se observa de la tabla de Análisis de Varianza que el P.adj < 0.05; por lo tanto,
se rechaza $H_0$; es decir, el tipo de tela teñida influye significativamente en el porcentaje de encogimiento de ella.

\item {Respecto a la hipótesis 2 (Factor B(Temperatura))}

Se observa de la tabla de Análisis de Varianza que el P.adj < 0.05; por lo tanto, se
rechaza $H_0$; es decir, los niveles de temperatura influyen significativamente en el porcentaje de encogimiento de la tela teñida.

\item {Respecto a la hipótesis 3 (Interacción(Tipo de tela y Temperatura))}

Se observa de la tabla de Análisis de Varianza que el P.adj < 0.05; por lo tanto, se
rechaza $H_0$; es decir, la combinación de el tipo de tela teñida y los niveles de temperatura
influyen en el porcentaje de encogimiento de la tela teñida.
\end{enumerate}

\section*{Diseño Trifactorial}
Estos tipos de Diseños experimentales son aquellos en los cuales se involucran en su
estudio tres factores; es decir, que se esta interesado en los efectos que producen los tres
factores en la variable respuesta en forma individual y conjunta (interacción). A los cuales se
les llama Diseños Trifactoriales.

Sean, A,B y C los factores que se van a estudiar en un experimento; el factor A tiene
“a” niveles, el factor B tiene “b” niveles y el factor C tiene “c” niveles; por lo tanto, cada
repetición del experimento tiene todas la “abc” combinaciones de tratamiento y en general hay
“n” repeticiones $(n = 2)$. El orden en que se toman las “abcn” observaciones en el experimento
debe ser aleatorio, de modo que este es un Diseño completamente aleatorizado.
Existen tres efectos principales (A,B y C), tres efectos dobles (AB,AC y BC) y un efecto
triple (ABC).

Los niveles de cada uno de los factores pueden ser elegidos de forma aleatoria, si es así
los factores son aleatorios o ser elegidos específicamente por el experimentador; es decir, los
que a él le interesan estudiar, entonces los factores son fijos.

Por lo tanto, de acuerdo a la forma en que son elegidos los niveles de los factores, así es
el Modelo que resulta y el Análisis de Varianza que se lleva a cabo.

\textbf{Ejemplo 2.}

En un experimento para investigar las propiedades de resistencia a la compresión de
mezclas de Cemento y Tierra, se utilizaron dos períodos (Edad A) diferentes de curado en
combinación con dos Temperaturas(B) diferentes de curado y dos tierras(C) diferentes. Se
hicieron dos réplicas para cada combinación de niveles de los tres factores, resultando los
siguientes datos:

\begin{table}[htb]
\centering
\begin{tabular}{||c|c|c|c|c||}
\hline
\hline
\multirow{4}{*}{Factor A (Edad)} & \multicolumn{4}{c||}{Factor B (Temperatura)} \\
\cline{2-5}
                    & \multicolumn{2}{c|}{1} & \multicolumn{2}{c||}{2} \\
\cline{2-5}
                    & \multicolumn{2}{c|}{Factor C (tierra)} & \multicolumn{2}{c||}{Factor C (tierra)}\\
\cline{2-5}

                   & 1 & 2 & 1 & 2 \\
\hline
\multirow{2}{*}{1} &471&385&485&530 \\
                   &413&434&552&593 \\
\hline
\multirow{2}{*}{2} &712&770&712&741 \\
                   &637&705&789&806 \\
\hline
\hline

\end{tabular}
\caption{Resistencia a la compresión de la mezcla de Cemento y Tierra}
\end{table}

\textbf{Solución}

\textbf{Planteamiento de las hipótesis a probar}

\textbf{Forma Estadística:}
\begin{enumerate}
  \item $H_0: \uptau_1 = \uptau_2 = 0$
  
        $H_1$:cuando menos un $\uptau_i \not= 0$                     
  \item $H_0:\beta_1 = \beta_2 = 0$ 
  
        $H_1$:cuando menos un $\beta_j \not= 0$                                   
  \item $H_0: \gamma_1 = \gamma_2 =0$  
  
        $H_1$:cuando menos un $\gamma_k \not= 0$                       
  \item $H_0: (\uptau\beta)_{ij} = 0$, para todo ij   
  
        $H_1$: cuando menos un $(\uptau\beta)_{ij} \not= 0$                   
  \item $H_0: (\uptau\gamma)_{ik} = 0$, para todo ik
  
        $H_1$: cuando menos un $(\uptau\gamma)_{ik} \not= 0$
  \item $H_0: (\beta\gamma)_{jk} = 0$, para todo jk
  
        $H_1$: cuando menos un $(\beta\gamma)_{jk} \not= 0$
  \item $H_0: (\uptau\beta\gamma)_{ijk} = 0$, para todo ijk
  
        $H_1$:cuando menos un  $(\uptau\beta\gamma)_{ijk} \not= 0$
\end{enumerate}

\textbf{Variable Respuesta:} Resistencia a la compresión de la mezcla de Cemento y Tierra.

\textbf{Forma Verbal}
\begin{enumerate}

\item $H_0$: La edad o períodos no influye significativamente en las propiedades de resistencia a la
compresión de mezclas de Cemento y Tierra.

      $H_1$: La edad o períodos influye significativamente en las propiedades de resistencia a la
compresión de mezclas de Cemento y Tierra.

\item $H_0$: La Temperatura no influye significativamente en las propiedades de resistencia a la
compresión de mezclas de Cemento y Tierra.

      $H_1$: La Temperatura influye significativamente en las propiedades de resistencia a la
compresión de mezclas de Cemento y Tierra.

\item $H_0$: Los tipos de tierra no influyen significativamente en las propiedades de resistencia a la
compresión de mezclas de Cemento y Tierra.

      $H_1$: Los tipos de tierra influyen significativamente en las propiedades de resistencia a la
compresión de mezclas de Cemento y Tierra.

\item $H_0$: La edad y la temperatura no influyen significativamente en las propiedades de
resistencia a la compresión de mezclas de Cemento y Tierra.

      $H_1$: La edad y la temperatura influyen significativamente en las propiedades de
resistencia a la compresión de mezclas de Cemento y Tierra

\item $H_0$: La edad y los tipos de tierra no influyen significativamente en las propiedades de
resistencia a la compresión de mezclas de Cemento y Tierra.

      $H_1$: La edad y los tipos de tierra influyen significativamente en las propiedades de
resistencia a la compresión de mezclas de Cemento y Tierra.

\item $H_0$: La temperatura y los tipos de tierra no influyen significativamente en las propiedades
de resistencia a la compresión de mezclas de Cemento y Tierra.

      $H_1$: La temperatura y los tipos de tierra influyen significativamente en las propiedades de
resistencia a la compresión de mezclas de Cemento y Tierra.

\item $H_0$: La edad, Temperatura y tipos de tierra no influyen significativamente en las
propiedades de resistencia a la compresión de mezclas de Cemento y Tierra.

      $H_1$: La edad, Temperatura y tipos de tierra influyen significativamente en las propiedades
de resistencia a la compresión de mezclas de Cemento y Tierra
\end{enumerate}

\begin{Schunk}
\begin{Sinput}
> library(xtable)
> #Leemos la hoja de datos
> datos<-read.csv("Ej2_F.csv", header = TRUE, sep = ";")
> #Analysis of Variance(aov) para 2 factores
> D3Fact<- aov(datos$OBS~ datos$Edad*datos$Temperatura*datos$Tierra)
> Resultado<-summary(D3Fact)
> #su equivalente
> Resultado2<- anova(D3Fact)
> #genarando el codigo latex para la tabla
> # de ANOVA
> Resultado3<- xtable(Resultado2) 
\end{Sinput}
\end{Schunk}

\begin{table}[ht]
\centering
\begin{tabular}{lrrrrr}
  \hline
 & Df & Sum Sq & Mean Sq & F value & Pr($>$F) \\ 
  \hline
Edad & 1 & 252255.06 & 252255.06 & 117.92 & 0.0000 \\ 
Temperatura & 1 & 28985.06 & 28985.06 & 13.55 & 0.0062 \\ 
Tierra & 1 & 2328.06 & 2328.06 & 1.09 & 0.3273 \\ 
Edad:Temperatura & 1 & 3393.06 & 3393.06 & 1.59 & 0.2434 \\ 
Edad:Tierra & 1 & 1425.06 & 1425.06 & 0.67 & 0.4380 \\ 
Temperatura:Tierra & 1 & 315.06 & 315.06 & 0.15 & 0.7111 \\ 
Edad:Temperatura:datos\$Tierra & 1 & 3335.06 & 3335.06 & 1.56 & 0.2471 \\ 
Residuals & 8 & 17113.50 & 2139.19 &  &  \\ 
   \hline
\end{tabular}
\caption{Análisis de Varianza}
\end{table}

\textbf{Conclusiones}

Se tiene:
\begin{enumerate}

\item {\textbf{Respecto a la Hipótesis 1 (Factor A(Edad))}}
Se observa de la tabla de Análisis de Varianza que el P.adj < 0.05; por lo tanto,
se rechaza $H_0$; es decir, la edad o períodos influye significativamente en las propiedades de
resistencia a la compresión de mezclas de Cemento y Tierra.

\item {\textbf{Respecto a la Hipótesis 2 (Factor B (Temperatura))}}
Se observa de la tabla de Análisis de Varianza que el P.adj < 0.05; por lo tanto,
se rechaza H0; es decir, la Temperatura influye significativamente en las propiedades de
resistencia a la compresión de mezclas de Cemento y Tierra.

\item {\textbf{Respecto a la Hipótesis 3 (Factor C (Tierra))}}
Se observa de la tabla de Análisis de Varianza que el P.adj > 0.05; por lo tanto, se
acepta H0; es decir, los tipos de tierra no influyen significativamente en las propiedades de
resistencia a la compresión de mezclas de Cemento y Tierra.

\item {\textbf{Respecto a la Hipótesis 4 (Interacción(Edad y Temperatura))}}
Se observa de la tabla de Análisis de Varianza que el P.adj > 0.05; por lo tanto, se
acepta H0; es decir, la edad y la temperatura no influye significativamente en las propiedades
de resistencia a la compresión de mezclas de Cemento y Tierra.

\item {\textbf{Respecto a la Hipótesis 5 (Interacción(Edad y Tierra))}}
Se observa de la tabla de Análisis de Varianza que el P.adj > 0.05; por lo tanto, se
acepta H0; es decir, la edad y los tipos de tierra no influyen significativamente en las
propiedades de resistencia a la compresión de mezclas de Cemento y Tierra.

\item {\textbf{Respecto a la Hipótesis 6 (Interacción(Temperatura y Tierra))}}
Se observa de la tabla de Análisis de Varianza que el P.adj > 0.05; por lo tanto, se
acepta H0; es decir, la temperatura y los tipos de tierra no influyen significativamente en las
propiedades de resistencia a la compresión de mezclas de Cemento y Tierra.

\item {\textbf{Respecto a la Hipótesis 7 (Interacción(Edad,Temperatura y Tierra))}}
Se observa de la tabla de Análisis de Varianza que el P.adj > 0.05; por lo tanto, se
acepta H0; es decir, la edad, Temperatura y tipos de tierra no influyen significativamente en
las propiedades de resistencia a la compresión de mezclas de Cemento y Tierra.

\end{enumerate}
\end{document}





