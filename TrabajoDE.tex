\documentclass[12pt,letterpaper]{report}
\usepackage[utf8]{inputenc}
\usepackage[spanish]{babel}
\usepackage{amsmath}
\usepackage{amsfonts}
\usepackage{amssymb}
\usepackage{graphicx}
\usepackage[left=3cm,right=2.5cm,top=3cm,bottom=2cm]{geometry}
\author{Fernando Manzanares}
\title{Diseños de Experimentos en R}
\renewcommand{\baselinestretch}{1.5}
\setlength{\parskip}{4mm}
\usepackage{ragged2e}
\justifying
\usepackage{longtable}
\usepackage{Sweave}
\setlength\parindent{0pt}
\usepackage{titlesec} 
\usepackage{xcolor}
\titleformat{\chapter}[display]{\bfseries\Huge}{\Huge\thechapter.}{0.5em}{}
\titlespacing{\chapter}{0pt}{0ex}{1pc}
\widowpenalty=10000  
\clubpenalty=10000

\begin{document}
\Sconcordance{concordance:TrabajoDE.tex:TrabajoDE.Rnw:%
1 29 1}
\Sconcordance{concordance:TrabajoDE.tex:./Portada2.Rnw:ofs 30:%
1 34 1}
\Sconcordance{concordance:TrabajoDE.tex:./Conceptos.Rnw:ofs 65:%
1 86 1}
\Sconcordance{concordance:TrabajoDE.tex:./D_unifact.Rnw:ofs 152:%
1 73 1 1 2 1 0 1 2 1 0 1 2 1 0 1 1 1 2 1 0 1 3 5 0 1 2 33 1 1 2 1 0 1 1 %
1 4 3 0 1 3 5 0 1 2 3 1 1 2 1 0 2 1 3 0 1 2 60 1 1 7 6 0 5 1 3 0 1 2 24 %
1 1 2 1 0 1 1 1 4 3 0 1 3 5 0 1 2 3 1 1 2 1 0 7 1 3 0 1 2 45 1 1 2 1 0 %
2 1 3 0 1 2 59 1 1 2 1 0 1 3 2 0 3 1 4 0 1 3 20 1 1 2 1 0 1 1 1 4 3 0 1 %
3 5 0 1 2 3 1 1 2 1 0 7 1 3 0 1 2 48 1 1 2 1 0 2 1 3 0 1 2 26 1}
\Sconcordance{concordance:TrabajoDE.tex:./D_Bloques.Rnw:ofs 687:%
1 92 1 1 2 1 0 1 2 1 0 1 2 1 0 1 1 1 2 1 0 1 3 5 0 1 2 24 1 1 2 1 0 1 1 %
1 4 3 0 1 3 5 0 1 2 1 1 1 2 1 0 7 1 3 0 1 2 49 1 1 2 1 0 2 1 3 0 1 2 81 %
1 1 2 1 0 1 2 1 0 1 2 1 0 1 1 1 2 1 0 1 3 5 0 1 2 23 1 1 2 1 0 1 1 1 4 %
3 0 1 3 5 0 1 2 1 1 1 2 1 0 7 1 3 0 1 2 48 1 1 2 1 0 2 1 3 0 1 2 22 1 1 %
2 17 0 1 2 50 1 1 2 1 0 1 4 3 0 1 4 3 0 2 1 3 0 1 2 17 1}
\Sconcordance{concordance:TrabajoDE.tex:./D_factoriales.Rnw:ofs 1233:%
1 147 1 1 2 1 0 1 2 1 0 1 2 1 0 1 1 1 2 1 0 1 3 5 0 1 2 112 1 1 2 1 0 1 %
3 2 0 1 4 3 0 2 1 3 0 1 2 154 1 1 2 1 0 1 2 1 0 1 2 1 0 1 1 1 2 1 0 1 3 %
5 0 1 2 61 1}
\Sconcordance{concordance:TrabajoDE.tex:TrabajoDE.Rnw:ofs 1755:%
36 5 1}

\begin{titlepage}

\begin{center}
\vspace*{-1in}
\begin{figure}[htb]
\begin{center}
\includegraphics[width=6cm]{Minerva}
\end{center}
\end{figure}

FACULTAD MULTIDISCIPLINARIA DE OCCIDENTE\\
\vspace*{0.15in}
DEPARTAMENTO DE MATEMÁTICAS \\
\vspace*{0.6in}
\begin{large}
TITULO:\\
\end{large}
\vspace*{0.2in}
\begin{Large}
\textbf{DISEÑOS DE EXPERIMENTOS EN R} \\
\end{Large}
\vspace*{0.3in}
\begin{large}
Fernando Ernesto Manzanares Morán\\
\end{large}
\vspace*{0.3in}
\rule{80mm}{0.1mm}\\
\vspace*{0.1in}
\begin{large}
Docente: \\
Salvador Enrique Rodriguez Hernandez \\
\end{large}
\end{center}

\end{titlepage}
\chapter*{Definición de un diseño experimental y conceptos básicos}
El \textbf{diseño de un experimento} es la secuencia completa de los pasos que se deben
tomar de antemano, para planear y asegurar la obtención de toda la información relevante y
adecuada al problema bajo investigación, la cual será analizada estadísticamente para obtener
conclusiones válidas y objetivas con respecto a los objetivos planteados.

Un \textbf{diseño experimental} es una prueba o serie de pruebas en las cuales existen
cambios deliberados en las variables de entrada de un proceso o sistema, de tal manera que
sea posible observar e identificar las causas de los cambios que se producen en la respuesta de
salida.

\textbf{Proposito de un diseño experimental}

El propósito de cualquier Diseño Experimental, es proporcionar una cantidad máxima de
información pertinente al problema que se está investigando. Y ajustar el diseño que sea lo
mas simple y efectivo; para ahorrar dinero, tiempo, personal y material experimental que se va
ha utilizar. Es de acotar, que la mayoría de los diseños estadísticos simples, no sólo son fáciles
de analizar, sino también son eficientes en el sentido económico y en el estadístico.
De lo anterior, se deduce que el diseño de un experimento es un proceso que explica
tanto la metodología estadística como el análisis económico.

\textbf{Conceptos básicos}


\textbf{Diseño:} Consiste en planificar la forma de hacer el experimento, materiales y métodos a
usar, etc.

\textbf{Experimento:} Conjunto de pruebas o ensayos cuyo objetivo es obtener 
información, que permita mejorar el producto o el proceso en estudio.

\textbf{Tratamiento:}
- Es un conjunto particular de condiciones experimentales definidas por el investigador; son el conjunto de circunstancias creadas por el experimento, en respuesta a la hipótesis de investigación y son el centro de la misma.

\textbf{Factor:} Es un grupo específico de tratamientos. (Ejemplo, Temperatura, humedad, tipos
de suelos, etc.).

\textbf{Niveles del factor:} Son diversas categorías de un factor. ( Por ejemplo, los niveles
de temperatura son 20°C, 30°C, etc.). Un factor Cuantitativo tiene niveles
asociados con puntos ordenados en alguna escala de medición, como
temperatura; mientras que los niveles de un factor cualitativo representan
distintas categorías o clasificaciones, como tipo de suelo, que no se puede acomodar
conforme a alguna magnitud.

\textbf{Réplica:} Son las repeticiones que se realizan del experimento básico.

\textbf{Unidad experimental:}

Es el material experimental unitario que recibe la aplicación de un tratamiento.

Es la entidad física o el sujeto expuesto al tratamiento independientemente de las
otras unidades. La unidad experimental una vez expuesta al tratamiento constituye
una sola réplica del tratamiento.

Es el objeto o espacio al cual se aplica el tratamiento y donde se mide y analiza la
variable que se investiga.

Es el elemento que se está estudiando.

\textbf{Unidad muestral:} Es una fracción de la unidad experimental que se utiliza 
para medir el efecto de un tratamiento.

\textbf{Error experimental:} Es una medida de variación que existe entre dos o 
más unidades experimentales, que han recibido la aplicación de un mismo tratamiento de
manera idéntica e independiente.

\textbf{Factores controlables:} Son aquellos parámetros o características del 
producto o proceso, para los cuales se prueban distintas variables o
valores con el fin de estudiar cómo influyen sobre los resultados.

\textbf{Factores incontrolables:} Son aquellos parámetros o características del 
producto o proceso, que es imposible de controlar al momento de
desarrollar el experimento.

\textbf{Variabilidad natural:} es la variación entre las unidades experimentales, 
que el experimentador no puede controlar ni eliminar.

\textbf{Variable dependiente:} es la variable que se desea examinar o estudiar en 
un experimento. (Variable Respuesta).

\textbf{Hipótesis:}

Es una suposición o conjetura que se plantea el investigador de una realidad
desconocida.

Es el supuesto que se hace sobre el valor de un parámetro (constante que caracteriza
a una población) el cual puede ser validado mediante una prueba estadística.

\newpage
\chapter*{Diseños unifactoriales}
En el análisis de los resultados de los experimentos se pueden observar diferentes
aplicaciones de los Diseños Experimentales. Hay experimentos muy útiles en los cuales existe
un sólo factor de interés; el cual se analiza por medio de la comparación de dos condiciones
que intervienen en el Experimento (a menudo llamadas tratamientos o niveles del factor); a
este tipo de experimentos se le denomina Experimentos de Comparación Simple,
estudiados en los cursos de Estadística básica. El análisis de los datos de este tipo de
Experimentos resulta ser sencillo, ya que se utilizan técnicas de la Inferencia Estadística,
llamada Prueba de Hipótesis (o pruebas de significancia) que son las que ayudan al
experimentador a comparar estas condiciones.

Si en el tipo de Diseño Experimental planteado anteriormente se requiere más de dos
niveles del factor que se analiza, éstos son considerados como "Diseños Unifactoriales".
Teniendo en cuenta que para el análisis de éstos se utiliza el Análisis de Varianza, ya que se
requiere probar la igualdad de varias medias.

En los experimentos de los Diseños Unifactoriales, el número de observaciones
recolectadas en cada tratamiento pueden ser iguales o diferentes. Cuando el número de
observaciones sea diferente se dice que el Diseño está Desequilibrado o Desbalanceado; en
caso contrario el Diseño está Equilibrado o Balanceado.

En este documento se hará uso del caso en el que el Diseño está Equilibrado o Balanceado y se utilizará un modelo de efectos fijos (en este tipo
de Modelos se desea probar hipótesis en relación a las medias de los tratamientos y las
conclusiones sólo se aplicarán a los niveles del factor considerados en el análisis, las
conclusiones no pueden extenderse a tratamientos similares que no se consideraron.) y no se profundizara mas teóricamente ya que el fin de este documento es la aplicación de código del software estadistico R para la aplicación de dicho modelo.

\textbf{Ejemplo 1.}

El Ministerio de Educación esta interesado en implementar tres programas de estudio; con el
objetivo de medir la habilidad de lectura en los alumnos. Para ello, se eligen alumnos del sexto
grado de un Colegio de San Salvador, de los cuales fueron asignados al azar 27 alumnos, a
cada uno de los tres grupos. Se utilizó un programa diferente en cada grupo, se llevó a cabo un
examen al inicio y al final de la implementación de los programas, los valores obtenidos
representan la diferencia que hay entre la nota del examen que se hizo al inicio y al final de la
implementación del programa, obteniéndose los siguientes datos, en base 100:

\begin{Schunk}
\begin{Sinput}
> datos<-read.csv("Ej1HL.csv", header = TRUE, sep = ";")
> unifact<- aov(HL~Programa, data = datos)
> summary(unifact)
\end{Sinput}
\begin{Soutput}
            Df Sum Sq Mean Sq F value Pr(>F)
Programa     2  36.22   18.11    1.44  0.257
Residuals   24 301.78   12.57               
\end{Soutput}
\end{Schunk}




\end{document}





